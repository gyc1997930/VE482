\documentclass{article}

\usepackage{indentfirst}
\usepackage{color}
\usepackage{graphicx}
\usepackage{amssymb}
\usepackage{amsmath}
\usepackage{geometry}
\usepackage{cite}
\usepackage{setspace}
\usepackage{drawmatrix}
\usepackage{listings}
\usepackage{xcolor}
\lstset{
    columns=fixed,
    numbers=left,
    frame=none,
    backgroundcolor=\color[RGB]{245,245,244},
    keywordstyle=\color[RGB]{40,40,255},
    numberstyle=\footnotesize\color{darkgray},
    commentstyle=\it\color[RGB]{0,96,96},
    stringstyle=\rmfamily\slshape\color[RGB]{128,0,0},
    showstringspaces=false,
    language=C++,
}
\geometry{left=2.54cm,right=2.54cm,top=2.54cm,bottom=2.54cm}
\begin{document}
\begin{spacing}{1.5}
\vspace*{0.25cm}

\hrulefill

\thispagestyle{empty}

\begin{center}
\begin{large}
\sc{UM--SJTU Joint Institute \vspace{0.3em} \\ Introduction to Operating Systems \\(VE482)}
\end{large}

\hrulefill

\vspace*{5cm}
\begin{Large}
\sc{{Homework 5}}
\end{Large}

\vspace{2em}

\end{center}


\vfill

\begin{table}[h!]
\flushleft
\begin{tabular}{lll}
Name: Ji Xingyou \hspace*{2em}&
ID: 515370910197\hspace*{2em}
\\

Date: 2 November 2017

\end{tabular}
\end{table}

\hfill

\newpage
\tableofcontents
\newpage
\section{\textit{Simple questions}}
\begin{enumerate}
	\item\textbf{A system has two processes and three identical resources. Each process needs a maximum of two resources. Can a deadlock occur? Explain.}\\
	No, deadlock does not occur here. The reason is that at least one process can get all the resources it requires. Then, this process will be executed and finished. Then, it releases the resources and passes them to the other process.
	\item\textbf{A computer has six tape drives, with n processes competing for them. Each process may need two drives. For which values of n is the system deadlock free?}\\
	If $n=3$, the distribution of tape drives will be $(2,2,2)$. Therefore, each process can have two drives to execute and then finish. No deadlock.\\
	If $n=4$, the distribution of tape drives will be $(2,2,1,1)$. Therefore, the first two processes will be able to execute and finish. When they are finished, they release the resources and pass them to other processes. No deadlock.\\
	If $n=5$, the distribution of tape drives will be $(2,1,1,1,1)$. Therefore, the first process will be able to execute and finish. This is similar to the previous case. No deadlock.\\
	If $n=6$, the distribution of tape drives will be $(1,1,1,1,1,1)$. All these six processes will compete for a tape drive owned by others. Therefore, in this case, a deadlock occurs.\\
	Hence, a deadlock occurs when $n>=6$.
	\item\textbf{A real-time system has four periodic events with periods of 50, 100, 200, and 250 msec each. Suppose the four events require 35, 20, 10, and x msec of CPU time, respectively. What is the largest value x for which the system is schedulable?}
	\begin{align}
	\frac{35}{50}+\frac{20}{100}+\frac{10}{200}+\frac{x}{250}&\leqslant1
	\end{align}
	Hence, the largest value for $x$ is $\frac{25}{2}$.
	\item\textbf{Round-robin schedulers normally maintain a list of all runnable processes, with each process oc- curring exactly once in the list. What would happen if a process occurred more than once in the list? Would there be any reason for allowing this?}\\

	\item\textbf{Can a measure of whether a process is likely to be CPU bound or I/O bound be detected by analyzing the source code. How to determine it at runtime?}\\

\end{enumerate}
\section{\textit{Deadlocks}}
\begin{enumerate}
	\item\textbf{Determine the content of the Request matrix.}\\
	\begin{equation}
	R=\left(           
  	\begin{array}{ccc}
    7 & 4 & 3\\
    1 & 2 & 2\\
    6 & 0 & 0\\
    0 & 1 & 1\\
    4 & 3 & 1\\
  	\end{array}
	\right)
	\end{equation}
	\item\textbf{Is the system in a safe state?}\\
	Yes.
	\item\textbf{Can all the processes be completed without the system being in an unsafe state at any stage?}
	\begin{equation}
	E=\left(           
  	\begin{array}{ccc}
    10 & 5 & 7\\
  	\end{array}
	\right)
	\end{equation}

	\begin{equation}
	A=\left(           
  	\begin{array}{ccc}
    3 & 3 & 2\\
  	\end{array}
	\right)
	\end{equation}

	\begin{equation}
	C=\left(           
  	\begin{array}{ccc}
    0 & 1 & 0\\
    2 & 0 & 0\\
    3 & 0 & 2\\
    2 & 1 & 1\\
    0 & 0 & 2\\
  	\end{array}
	\right)
	\end{equation}

	\begin{equation}
	R=\left(           
  	\begin{array}{ccc}
    7 & 4 & 3\\
    1 & 2 & 2\\
    6 & 0 & 0\\
    0 & 1 & 1\\
    4 & 3 & 1\\
  	\end{array}
	\right)
	\end{equation}
	We can assume that the system first allocates the available resources to $P_{2}$, then $P_{4}$, $P_{5}$, $P_{3}$ and $P_{1}$. No deadlock occurs in this order.
\end{enumerate}
\newpage
\section{\textit{Research\cite{ex3}}}
\textbf{Write about a page on the topic of viruses, worms and Trojans; In particular explain what they are, how they are created, and how to avoid them. Do not forget to reference your sources of information.}
\subsection{\textit{Viruses}}
According to the \textit{Computer System Safety Protection Regulations of the People's Republic of China}, the viruses are defined as ``a set of computer instructions or programming codes that is inserted in the computer programs, breaking computer functions or data, and is able to duplicate itself''. There are two requirements that viruses must meet.
\begin{enumerate}
	\item It must execute automatically. In common cases, it hides itself under the path of another program.
	\item It must be able to duplicate itself. 
\end{enumerate}
Meanwhile, viruses have the features of strong infectious, certain latency, specific trigger and huge destructiveness.
\subsection{\textit{Worms}}
Worms can be regarded as one specific type of viruses, but it has huge differences between normal virus. It is generally accepted that worm is one kind of virus spread through the Internet. Worm has some common features with virus, such as transmissibility, imperceptibility and destructiveness. Also, worm has some features that only belong to itself. For example, it does not rely on files to parasite, causes refusal service to the Internet and can be combined with the hack tech. Compared to virus, which relies on the infected file to duplicate, worm can duplicate itself among systems spontaneously. Also, different from virus, which only attacks the files in the computer, worm can attack all the targeted computers within the Internet. Once your computer is infected with worm, it can quickly spread to many other computers connected to the same Internet you are using and duplicate incredibly. Therefor, from the aspect of destructiveness, worm is much powerful than virus.
\subsection{\textit{Trojans}}
Trojan refers to the program that seems to be useful, but turns out to be destructive. It is a deceptive hacking tool that is based on remote control technology. It has the feature of imperceptibility and unauthorization. Unlike virus and worm, which may duplicate itself and intentionally infect other files, Trojan hides itself into a nice-looking file and lurks the user to download or execution. Once your computer is infected with Trojan, your system may be totally open to the hacker.
\section{\textit{Programming}}
\textbf{Implement the Banker’s algorithm.}\\
See the attached ex4.cpp file and the Readme.txt for explanation.
\section{\textit{Minix 3}}
\textbf{How is scheduling handled in Minix 3? Provide clear explanations on how to find the information just by exploring the source code of Minix kernel.}\\

\section{\textit{The reader-writer problem}}
\begin{enumerate}
	\item\textbf{Explain how to get a read lock, and write the corresponding pseudocode.}\\
	
	\item\textbf{Describe what is happening if many readers request a lock.}\\

	\item\textbf{Explain how to implement this idea using another semaphore called read lock.
}\\

	\item\textbf{Is this solution giving any unfair priority to the writer or the reader? Can the problem be considered as solved?}

\end{enumerate}
\begin{thebibliography}{0}
\bibitem{ex3}
Kairry, \textit{The differences among Viruses, Worms and Trojans}, 2010-10-20, http://www.360doc.com/content/10/1020/18/2630460\_62522631.shtml
\end{thebibliography}
\end{spacing}
\end{document}
